\input{article preambles}

\setcounter{section}{-1}

\input{commands}

%I usually write \tensor instead of \otimes

\newcommand{\toroidal}{ \tilde{\g}_{[2]} } %toroidal lie (bi)algebras (without derivations)
\newcommand{\extendedtoroidal}{ \hat{\g}_{[2]} } %toroidal lie (bi)algebras (with derivations)
\renewcommand{\simpleroots}{\mathbb{I}}
\newcommand{\divzero}{\d_{[2]}}

\begin{document}

    \title{Classical limits of affine Yangians}
    
    \author{Dat Minh Ha and Peggy Jankovic}
    \maketitle
    
    \begin{abstract}
        We construct a topological Lie bialgebra structure on toroidal Lie algebras equipped with a degree-$(0, -1)$ bilinear form - as in \cite{msc_thesis_gamma_extended_toroidal_lie_algebras} - and then verify that, in the cases when the \say{underlying} affine Kac-Moody algebra is of a simply laced type, this coincides with the classical limits of the corresponding affine Yangians. Additionally, we make some connections with the theory of EALAs in the multi-loop realisation from \cite{allison_berman_faulkner_pianzola_multiloop_realisation_of_EALAs}, as well as that of quantum toroidal algebra from \cite{laurie_toroidal_QUEs_coproducts_2}.
    \end{abstract}
    
    {
    \hypersetup{} 
    %\dominitoc
    \tableofcontents %sort sections alphabetically
    }

    \section{Introduction}
    \subsection{Notations}
        \begin{convention}[Formal distributions]
            Given any formal distribution:
    $$a(z) := \sum_{m \in \Z} a_m z^{-m} \in V[\![z^{\pm 1}]\!]$$
with coefficients in some vector space $V$, let us write:
    $$a(z)^+ := \sum_{m \geq 0} a_m z^{-m} \quad, \quad a(z)^- := \sum_{m < 0} a_m z^{-m}$$

In order to avoid confusion with cobrackets, which are usually denoted by $\boldsymbol{\delta}$, we use:
    $$\1(z) := \sum_{m \in \Z} z^{-m} \in \bbk[\![z^{\pm 1}]\!]$$
to mean the formal Dirac distribution. Note that:
    \begin{equation} \label{equation: positive_negative_dirac_delta_distributions}
        \begin{gathered}
            \1(z)^+ = \sum_{m \geq 0} z^{-m} = \frac{1}{1 - z^{-1}}
            \\
            \1(z)^- = \sum_{m < 0} z^{-m} = -1 + \sum_{m \geq 0} z^m - 1 = -1 + \frac{1}{1 - z} = \frac{z}{1 - z}
        \end{gathered}
    \end{equation}
In particular, note that for $z = t_2/t_1$, we have:
    $$
        \begin{gathered}
            \1(t_2/t_1)^+ = \frac{1}{1 - (t_2/t_1)^{-1}} = \frac{1}{1 - t_1/t_2} = \frac{t_2}{t_2 - t_1}
            \\
            \1(t_2/t_1)^- = \frac{t_2/t_1}{1 - t_2/t_1} = \frac{t_2}{t_1 - t_2} = -\frac{t_2}{t_2 - t_1}
        \end{gathered}
    $$
        \end{convention}
    
        \begin{convention}
            Throughout, we fix a finite-dimensional simple Lie algebra $\g$ over $\bbC$, though in principle, we can work over any algebraically closed field characteristic $0$. On this Lie algebra, we choose once and for all a symmetric, non-degenerate, and invariant bilinear form $(\cdot, \cdot)_{\g}$; \textit{a priori}, any such bilinear form is a non-zero scalar multiple of the Killing form. Let us fix also a Cartan subalgebra $\h \subset \g$ of dimension $l$. By means of the bilinear form fixed above, we can define a root system $\rootsystem \subset \h^*$, in which we choose a subset of simple roots indexed by a set $\simpleroots$ of cardinality $l$. With respect to these choices, we have a set of Chevalley-Serre generators $\{h_i, e_i^{\pm}\}_{i \in \simpleroots}$, normalised so that $(h_i, h_j)_{\g} = \1_{i, j}$, wherein $\1_{i, j}$ is the Kronecker delta.
        \end{convention}

        \todo[inline]{Dat: I honestly cannot decide if $\boldsymbol{ \caly }, \boldsymbol{ \calu }$, or $\fraku$, or $\t$ is the best notation for toroidal Lie algebras. The first is supposed to invoke "Yangian", but it is slightly misleading since the affine Yangian descends to $\boldsymbol{ \caly }^+$ (the "positive" part of the Manin triple). The second and third one should probably be reserve for the classical limit of the quantum toroidal algebra. The last one is the one that Curtis uses, but my reservations with it are that 1. $\t$ is also usually used for toral Lie subalgebras (in the sense of being a Lie subalgebra containing semi-simple elements), 2. we already have a lot of t's, and 3. $\t^+$ is kind of an eye sore in my opinion, since it consists of two similar symbols. What do you think ?}

    \subsection{Context}

    \subsection{Overview}

    \section{Toroidal Lie (bi)algebras}
    \subsection{(Extended) toroidal Lie algebras}
        Let us begin by recalling the following description of universal central extensions (UCEs) of current algebras, i.e. of Lie algebras of the form $\g \tensor A$ wherein $A$ is a commutative algebra, due to Kassel. For a self-contained but comprehensive account of the following facts, we direct the reader to \cite[Section 2.2]{msc_thesis_gamma_extended_toroidal_lie_algebras}.
        
        First of all, for brevity, let us write $xf$ to denote the pure tensors $x \tensor f \in \g \tensor A$. On $\g \tensor A$, the Lie algebra structure is given by:
            \begin{equation} \label{equation: current_algebra_brackets}
                [xf, yg]_{\g \tensor A} := [x, y]_{\g} fg
            \end{equation}
        It is worth noting, that with this Lie bracket, $\g \tensor A$ is a perfect Lie algebra\footnote{A Lie algebra is said to be perfect if it is equal to its derived subalgebra.} as a consequence of the simplicity of the Lie algebra $\g$ that we assumed from before. This fact is important because Lie algebras are perfect if and only if they admit UCEs (see \cite[Lemma 1.10]{garland_arithmetics_of_loop_groups}); for a perfect Lie algebra $\a$, we denote its UCE by $\uce(\a)$. It is also desirable to have a description of the structure of $\toroidal_A$, and since:
            $$\toroidal_A := \uce(\g \tensor A) \cong ( \g \tensor A ) \oplus^{\e} \z_A$$
        wherein $\e: \bigwedge^2 (\g \tensor A) \to \z_A$ is the universal Lie $2$-cocycle with values in the centre $\z_A := \z( \toroidal_A )$, it suffices to give a more detailed description of the centre. In \cite{kassel_universal_central_extensions_of_lie_algebras}, Kassel demonstrated via the use of cyclic cohomology, that the centre can be given via:
            $$\z_A \cong \bar{\Omega}^1_A := \Omega^1_A/dA \quad, \quad \e(xf, yg) := (x, y)_{\g} g \bar{d}f$$
        wherein $\Omega^1_A$ denotes the $A$-module of algebraic $1$-forms on $A$, regarded here as a $\bbC$-vector space, $d: A \to \Omega^1_A$ is the initial object in the category of $A$-module homomorphisms $d': A \to \Omega'$ such that $d'(fg) = fdg + gdf$ (see \cite[\href{https://stacks.math.columbia.edu/tag/00RM}{Tag 00RM}]{stacks-project} for more details), and we have written:
            $$\bar{d}f \in \bar{\Omega}^1_A$$
        for the image of $df \in \Omega^1_A$ under the canonical quotient map $\Omega^1_A \to \bar{\Omega}^1_A$. In particular, this implies that the non-trivial commutators in $\toroidal_A$ are given by the following formula:
            \begin{equation} \label{equation: toroidal_lie_algebra_commutators}
                \begin{aligned}
                    [xf, yg]_{\toroidal_A} & = [xf yg]_{\g \tensor A} + \e(xf, yg)
                    \\
                    & = [x, y]_{\g} fg + (x, y)_{\g} g \bar{d}f
                \end{aligned}
                \quad, \quad
                x, y \in \g, f, g \in A
            \end{equation}

        More can be said when $A$ is graded by some abelian group $Z$, say $A := \bigoplus_{n \in Z} A_n$. In this case, $\g \tensor A$ and $\Omega^1_A$ carry an induced $Z$-grading given by:
            $$\g \tensor A = \bigoplus_{n \in Z} (\g \tensor A)_n \quad, \quad (\g \tensor A)_n := \g \tensor A_n$$
            $$\Omega^1_A = \bigoplus_{n \in Z} (\Omega^1_A)_n \quad, \quad (\Omega^1_A)_n := \bigoplus_{ \substack{i, j \in Z\\i + j = n} } \frac{A_i \tensor A_j}{ \< \forall x_i \in A_i, x_j \in A_j: x_i x_j \tensor 1 - x_i \tensor 1 - 1 \tensor x_j = 0 \> }$$
        Together, these two induced $Z$-gradings induce a further $Z$-grading on $\toroidal_A$, determined by:
            $$\bar{\Omega}^1_A := \bigoplus_{n \in Z} ( \bar{\Omega}^1_A )_n, \quad, \quad ( \bar{\Omega}^1_A )_n := ( \Omega^1_A )_n/dA_n$$
        At the same time, for a distinguished element $n_0 \in Z$, one can define a symmetric, non-degenerate, and invariant bilinear form on $\g \tensor A$ given by:
            \begin{equation} \label{equation: current_algebra_pairing}
                (x f, y g)_{\g \tensor A} := (x, y)_{\g} \delta_{ \deg f + \deg g, n_0 }
            \end{equation}
        This form extends naturally to $\toroidal_A$, though by invariance, this extension does not retain the non-degeneracy property: in fact, it is easy to show that the radical of this extended bilinear form - which shall be denoted by $(\cdot, \cdot)_{\toroidal_A}$ - is precisely the centre $\z_A$.

        The fact that the bilinear form $(\cdot, \cdot)_{\toroidal_A}$ is no longer non-degenerate prompts us to enlarge $\toroidal_A$ by a vector space $\d_A$, i.e. we form:
            $$\extendedtoroidal_A := \toroidal_A \oplus \d_A$$
        in such a way that the extension of the bilinear form $(\cdot, \cdot)_{\toroidal_A}$ to $\extendedtoroidal_A$, which shall be denoted by:
            $$(\cdot, \cdot)_{\extendedtoroidal_A}$$
        becomes non-degenerate. This means that the bilinear form $(\cdot, \cdot)_{\extendedtoroidal_A}$ is to pair $\d_A$ non-degenerately with $\z_A$, and in order to construct such a bilinear form, let us take:
            $$\d_A := \z_A^{\star}$$
        where $\z_A^{\star} := \bigoplus_{(r, s) \in \Z^2} (\z_A)_{(r, s)}^*$ is the graded dual.
        \begin{lemma}[A non-degenerate bilinear form extending $(\cdot, \cdot)_{\toroidal}$] \label{lemma: extended_toroidal_bilinear_form}
            On the vector space $\extendedtoroidal_A$, there is a non-degenerate and symmetric bilinear form $(\cdot, \cdot)_{\extendedtoroidal_A}$ given by:
                $$( X + D, Y + D' )_{\extendedtoroidal_A} := (X, Y)_{\toroidal_A} + D'( \pi_{\z_A}(X) ) + D( \pi_{\z_A}(Y) )$$
            for all $X, Y \in \toroidal_A$ and all $D, D' \in \d_A$, extending the bilinear form $(\cdot, \cdot)_{\toroidal_A}$ on $\toroidal_A$. Here, $\pi_{\z_A}: \toroidal_A \to \z_A$ denotes the canonical projection.
        \end{lemma}
        
        In \cite{msc_thesis_gamma_extended_toroidal_lie_algebras}, it was demonstrated that this data is enough to specify a family of Lie algebra structures on $\extendedtoroidal_A$, parametrised by (abelian) Lie $2$-cocycles:
            $$\sigma \in H^2_{\Lie}(\d_A, \z_A)$$
        We care in particular about cocycles:
            $$\sigma \equiv 0$$
        corresponding to the semi-direct product:
            $$\toroidal_A \rtimes \d_A$$
        and henceforth, this shall be the Lie algebra structure that we shall equip $\extendedtoroidal_A$ with. Now, we should also mention that, due to the fact that $\z_A \cong \bar{\Omega}^1_A$, one can identify $\d_A$ with a Lie subalgebra of the Lie algebra $\der(A)$ of derivations on $A$; it is also possible to compute an explicit basis for $\d_A$. Which Lie subalgebra $\d_A$ is isomorphic to in particular, though, depends heavily on the degree of $(\cdot, \cdot)_{\extendedtoroidal_A}$. Consequently, how the Lie brackets between basis elements of $\extendedtoroidal$ are given also depends on the degree of this bilinear form.
        \begin{remark}
            In general, the Lie algebra structures on $\extendedtoroidal_A$ are \say{twists} of the Lie bracket on the semi-direct product $\toroidal_A \rtimes \d_A$. In \cite{msc_thesis_gamma_extended_toroidal_lie_algebras}, these were denoted by $\toroidal_A \rtimes^{\sigma} \d_A$, and they admit $\toroidal$ as a Lie ideal (and hence Lie subalgebra, in particular) regardless of $\sigma$.
        \end{remark}

        For us, the cases of particular interest are the following. From now on, let:
            $$A := \bbC[v^{\pm 1}, t^{\pm 1}]$$
        We shall let the grading abelian group be:
            $$Z := \Z^2$$
        and the cases that we shall consider are when the distinguished element $n_0 := (r_0, s_0) \in \Z^2$ are:
            $$(r_0, s_0) = (0, -1) \quad, \quad \text{and} \quad, \quad (r_0, s_0) = (0, 0)$$
        The reason for this is that in these cases, $\toroidal_A$ is to be the classical limits of - respectively - the centrally extended double Yangian associated to the affine Kac-Moody algebra $\hat{\g}$ associated to $\g$ (i.e. the Lie affinisation of $\g$) and the quantum toroidal algebra associated to $\g$ (i.e. the quantum affinisation associated to $\hat{\g}$). We expect the latter to somehow \say{degenerate} to the former, so it is important to analyse the relationship between their classical limits as well.

        Before we can do that, though, we must endow $\toroidal_A$ with a Lie bialgebra structure. There is, however, an obstacle: the theory of Manin triples (see e.g. \cite{etingof_kazhdan_quantisation_1}) tells us that it is necessary to have a non-degenerate, symmetric, and invariant bilinear form, yet we know that $(\cdot, \cdot)_{\toroidal_A}$ is \textit{degenerate}. That said, we do have a non-degenerate, symmetric, and invariant bilinear form on the larger Lie algebra $\extendedtoroidal_A$, namely $(\cdot, \cdot)_{\extendedtoroidal_A}$, meaning that we can construct a Lie bialgebra structure on $\extendedtoroidal_A$ instead. We shall then show in subsection \ref{subsection: toroidal_lie_bialgebras} that $\toroidal_A$ is a Lie sub-bialgebra of $\extendedtoroidal_A$.

        Let us now recall the explicit structures of $\extendedtoroidal_A$ in the two cases mentioned above. Our main reference is \cite{msc_thesis_gamma_extended_toroidal_lie_algebras}. First of all, let us describe a specific basis for $\z_A$. This was first done in \cite{moody_rao_yokonuma_vertex_representations_of_toroidal_lie_algebras}, but as many details were omitted in \textit{loc. cit.} (presumably because the material is well-known to experts), and for notational consistency, let us follow the presentation in \cite[Subsection 2.3.3]{msc_thesis_gamma_extended_toroidal_lie_algebras}. Through the induced $\Z^2$-grading on $\Omega^1_A$, and hence also on $\bar{\Omega}^1_A$, one can write:
            $$\z_A \cong \bigoplus_{(r, s) \in \Z^2} \bbC K_{r, s} \oplus \bbC c_v \oplus \bbC c_t$$
        wherein the basis elements are given by:
            \begin{equation} \label{equation: toroidal_centre_basis}
                \begin{gathered}
                    K_{r, s} :=
                    \begin{cases}
                        \frac1s v^{r - 1} t^s \bar{d}v & \text{if $(r, s) \in \Z \x (\Z \setminus \{0\})$}
                        \\
                        -\frac1r v^r t^{-1} \bar{d}t & \text{if $(r, s) \in (\Z \setminus \{0\}) \x \{0\}$}
                        \\
                        0 & \text{if $(r, s) = (0, 0)$}
                    \end{cases}
                    \\
                    c_v := v^{-1} \bar{d}v \quad, \quad c_t := t^{-1} \bar{d}t
                \end{gathered}
            \end{equation}
        In particular, this means that:
            $$\deg K_{r, s} = (r, s) \quad, \quad \deg c_v = \deg c_t = (0, 0)$$
        Moreover, any element of the form:
            $$v^n t^q \bar{d}(v^m t^p) \in \z_A$$
        can be written in terms of the basis vectors $K_{r, s}, c_v, c_t$ in the following manner:
            $$v^n t^q \bar{d}(v^m t^p) = (mq - np) K_{m + n, p + q} + \delta_{(m, p) + (n, q), (0, 0)} ( m c_v + p c_t )$$
        (cf. \cite[p. 35]{wendlandt_formal_shift_operators_on_yangian_doubles}).
        \begin{remark}[The centre of $\extendedtoroidal_A$]
            Let us caution the reader, that $\z_A$ is in fact \textit{not} the centre of the extended Lie algebra $\extendedtoroidal_A$. Rather, the centre of $\extendedtoroidal_A$ is considerably smaller. In particular, when $A = \bbC[v^{\pm 1}, t^{\pm 1}]$, it can be shown that the centre of the extended toroidal Lie algebra $\extendedtoroidal_A$ is given by:
                $$\z( \extendedtoroidal_A ) = \bbC c_v \oplus \bbC c_t$$
            That it is actually just $2$-dimensional is in good analogy with the $1$-dimensional centres of affine Kac-Moody algebras (see \cite[Chapter 7]{kac_infinite_dimensional_lie_algebras}).
        \end{remark}
        
        Now that we have an explicit description of the graded components of $\z_A$, we can compute the graded dual space $\d_A$. Let us emphasise again, that this depends heavily on the previously chosen bilinear form $(\cdot, \cdot)_{\toroidal_A}$. In particular, the non-degeneracy of the extension $(\cdot, \cdot)_{\extendedtoroidal_A}$ gives us a specific (graded) isomorphism of degree $(r_0, s_0) \in \Z^2$:
            $$\z_A = \bigoplus_{(r, s) \in \Z^2} (\z_A)_{r, s} \xrightarrow[]{\cong} \bigoplus_{(r, s) \in \Z^2} (\d_A)_{-r + r_0, -s + s_0} = \d_A$$
        and our reason for shifting the grading by $(r_0, s_0)$ is so that when we regard each restriction:
            $$(\cdot, \cdot)_{\extendedtoroidal_A}|_{(\z_A)_{r, s} \x (\d_A)_{-r + r_0, -s + s_0}}$$
        as a $2$-tensor, it will be of degree $(r_0, s_0)$, thus ensuring that the entire bilinear form $(\cdot, \cdot)|_{\extendedtoroidal_A}$ is of degree $(r_0, s_0)$ when regarded as a $2$-tensor on $\extendedtoroidal$ (note that the bilinear form \eqref{equation: current_algebra_pairing} is already of degree $(r_0, s_0)$ by construction).
        
        As alluded to earlier, since each graded component $(\z_A)_{(r, s)}$ is spanned by certain isomorphism classes of $1$-forms, it is reasonable to suspect that each dual graded component $(\z_A)_{(r, s) + (r_0, s_0)}^*$ is spanned by vector fields belonging to some Lie subalgebra of $\der(A)$, and indeed, this is the case. To this end, we introduce the following terminology.
        \begin{definition}[$\gamma$-divergence-zero vector fields] \label{def: div_zero_vector_fields}
            Let $\divzero(A)$ be the vector subspace of $\der(A)$ defined as follows:
                $$\divzero(A) := \{ D \in \der(A) \mid \forall f \in A: \gamma(D(f)) = 0 \}$$
            wherein $\gamma: A \to \bbC$ is the linear map given by $\gamma(f(v, t)) := -\Res( v^{-1 - r_0} t^{-1 - s_0} f(v, t) )$, with the residue being understood as the formal residue at $(v, t) = (0, 0)$, which extracts the coefficient of degree $(-1, -1)$. Elements of this vector space shall be referred to as \textbf{$\gamma$-divergence-zero vector fields}.
        \end{definition}
        \begin{remark}
            Note that the linear map $\gamma: A \to \bbC$ helps us recover the bilinear form \eqref{equation: current_algebra_pairing} in the following manner:
                $$(xf, yg)_{\g \tensor A} = (x, y)_{\g} \gamma(fg) \quad, \quad x, y \in \g, f, g \in A$$
        \end{remark}

        \begin{lemma} \label{lemma: div_zero_vector_fields_basic_properties}
            The subspace $\divzero(A) \subset \der(A)$ from definition \ref{def: div_zero_vector_fields} enjoys the following basic properties.
            \begin{enumerate}
                \item $\divzero(A)$ is a Lie subalgebra of $\der(A)$ (with the usual commutator bracket).
                \item $\divzero(A)$ admits the following subset as a basis:
                    $$\{D_{r, s}\}_{(r, s) \in \Z^2} \cup \{D_v, D_t\}$$
                Its elements are given in terms of the partial derivatives $\del_{v} := \frac{\del}{\del v}$ and $\del_{t} := \frac{\del}{\del t}$ by:
                    \begin{equation} \label{equation: div_zero_vector_fields_basis}
                        \begin{gathered}
                            D_{r, s} := -s v^{-r + r_0 + 1} t^{-s + s_0} \del_{v} + r v^{-r + r_0} t^{-s + s_0 + 1} \del_{t}
                            \\
                            D_v := -v^{r_0 + 1} t^{s_0} \del_{v} \quad, \quad D_t := -v^{r_0} t^{s_0 + 1} \del_{t}
                        \end{gathered}
                    \end{equation}
                \item The basis elements of $\divzero(A)$ satify the following commutation relations:
                    \begin{equation} \label{equation: div_zero_vector_fields_commutators}
                        \begin{gathered}
                            [D_v, D_t] = -D_{r_0, -s_0}
                            \\
                            [D_v, D_{r, s}] = r D_{r - r_0, s - s_0} \quad, \quad [D_t, D_{r, s}] = s D_{r - r_0, s - s_0}
                            \\
                            [D_{a, b}, D_{r, s}] = (br - as) D_{a + r - r_0, b + s - s_0}
                        \end{gathered}
                    \end{equation}
            \end{enumerate}
        \end{lemma}
            \begin{proof}
                Cf. \cite[Lemma 3.2.1.1]{msc_thesis_gamma_extended_toroidal_lie_algebras}.
            \end{proof}
        \begin{corollary}[$\Z^2$-grading on $\gamma$-divergence-zero vector fields] \label{coro: extended_toroidal_bilinear_forms}
            $\divzero(A)$ is a $\Z^2$-graded Lie subalgebra of the $\Z^2$-graded Lie algebra $\der(A)$ (the grading on the latter is the standard one coming from $A$). Namely, the grading on $\divzero(A)$ is given by:
                \begin{equation} \label{equation: div_zero_vector_fields_are_graded}
                    \begin{gathered}
                        \deg D_{r, s} = (-r + r_0, -s + s_0) \quad, \quad (r, s) \in \Z^2 \setminus \{(0, 0)\}
                        \\
                        \deg D_v = \deg D_t = (r_0, s_0)
                    \end{gathered}
                \end{equation}
        \end{corollary}
        \begin{proposition} \label{prop: div_zero_vector_fields_are_graded_dual_to_toroidal_centre}
            There is a $\Z^2$-graded vector space isomorphism:
                \begin{equation} \label{equation: div_zero_vector_fields_are_graded_dual_to_toroidal_centre}
                    \varphi: \divzero(A) \xrightarrow[]{\cong} \d_A
                \end{equation}
            given by:
                $$\varphi(D)( g \bar{df} ) := \gamma( g D(f) ) \quad, \quad D \in \divzero(A), f, g \in A$$
        \end{proposition}
            \begin{proof}
                Cf. \cite[Proposition 3.2.1.1]{msc_thesis_gamma_extended_toroidal_lie_algebras}.
            \end{proof}
        \begin{corollary}
            \begin{enumerate}
                \item Equation \eqref{prop: div_zero_vector_fields_are_graded_dual_to_toroidal_centre} identifies the basis $\{D_{r, s}\}_{(r, s) \in \Z^2} \cup \{D_v, D_t\}$ of $\divzero(A)$ (cf. equation \eqref{equation: div_zero_vector_fields_basis}) as being $\Z^2$-graded dual to the basis $\{K_{r, s}\}_{(r, s) \in \Z^2} \cup \{c_v, c_t\}$ of $\z_A$ (cf. equation \eqref{equation: toroidal_centre_basis}).
                \item Moreover, $\d_A$ can be endowed with a Lie algebra structure, given by the same formulae as in \eqref{equation: div_zero_vector_fields_commutators}.
                \item Lastly, there is indeed a non-degenerate symmetric bilinear form $(\cdot, \cdot)_{\varphi}$ on the vector space $\z_A \oplus \divzero(A)$, given by:
                    $$(K, D)_{\varphi} := \varphi(D)(K)$$
                    $$(K, K')_{\varphi} = (D, D')_{\divzero(A)} := 0$$
                for all $K, K' \in \z_A, D, D' \in \divzero(A)$. This extends to a non-degenerate and symmetric bilinear form $(\cdot, \cdot)_{\toroidal_A \oplus \divzero(A)}$ on $\toroidal_A \oplus \divzero(A)$, given by:
                    $$
                        \begin{aligned}
                            ( X + D, Y + D' )_{\toroidal_A \oplus \divzero(A)} & := (X, Y)_{\toroidal} + ( \pi_{\z_A}(X), D' )_{\varphi} + ( \pi_{\z_A}(Y), D )_{\varphi}
                            \\
                            & = (X, Y)_{\toroidal} + \varphi(D')( \pi_{\z_A}(X) ) + \varphi(D)( \pi_{\z_A}(Y) )
                        \end{aligned}
                    $$
                for all $X, Y \in \toroidal$ and all $D, D' \in \divzero(A)$, and with $\pi_{\z_A}: \toroidal_A \to \z_A$ denoting the canonical projection. This bilinear form itself extends the bilinear form $(\cdot, \cdot)_{\toroidal}$ on $\toroidal$, and it is related to the bilinear form $(\cdot, \cdot)_{\toroidal_A \oplus \d_A}$ from lemma \ref{lemma: extended_toroidal_bilinear_form} by:
                    $$(\cdot, \cdot)_{\toroidal_A \oplus \divzero(A)} = (\cdot, \cdot)_{\toroidal_A \oplus \d_A} \circ ( \id_{\toroidal} \oplus \varphi )^{\tensor 2}$$
            \end{enumerate}
        \end{corollary}

        \begin{proposition}[A $\d_A$-action on $\toroidal_A$ by Lie derivatives] \label{prop: vector_field_action_on_toroidal_lie_algebras}
            There is a $\der(A)$-module structure on $\toroidal_A$:
                $$\rho: \der(A) \to \der(\toroidal_A)$$
            by Lie derivations, given for all $D \in \der(A)$ by the following formulae:
                \begin{equation} \label{equation: vector_field_action_on_toroidal_lie_algebras}
                    \begin{gathered}
                        \rho(D)( xf ) := x D(f) \quad, \quad x \in \g, f \in A
                        \\
                        \rho( g\bar{d}f ) := D(g) \bar{d}f + g \bar{d}(D(f)) \quad, \quad f, g \in A
                    \end{gathered}
                \end{equation}
            By restricting this action to $\d_A$, one obtains an action of $\d_A$ on $\toroidal_A$ by derivations, given by the same formulae.
        \end{proposition}
            \begin{proof}
                See \cite[Proposition 3.2.4.2]{msc_thesis_gamma_extended_toroidal_lie_algebras}.
            \end{proof}
        \begin{corollary}[Existence of extended toroidal Lie algebras] \label{coro: existence_of_extended_toroidal_lie_algebras}
            The semi-direct product:
                $$\extendedtoroidal_A := \toroidal_A \rtimes \d_A$$
            exists. Consequently, there exists a section\footnote{... that is a Lie algebra homomorphism, not just a linear map.} $\d_A \to \extendedtoroidal_A$ of the canonical quotient map $\extendedtoroidal_A \to \d_A$, identifying $\d_A \subset \der(A)$ as a Lie subalgebra of $\extendedtoroidal_A$; in particular, this means:
                $$[D, D']_{\extendedtoroidal_A} = [D, D']_{\der(A)} = DD' - D'D, \quad, \quad D, D' \in \d_A$$
        \end{corollary}

        Using lemma \ref{lemma: div_zero_vector_fields_basic_properties} in conjunction with proposition \ref{prop: vector_field_action_on_toroidal_lie_algebras}, we can now also explicitly compute the commutation relations between the basis elements of $\d_A$ and $\z_A$.
        \begin{lemma} \label{lemma: explicit_commutators_between_central_basis_elements_and_derivations}
            In the Lie algebra $\extendedtoroidal_A$, one has the following commutation relations between elements of $\z_A$ (cf. equation \eqref{equation: toroidal_centre_basis}) and those of $\d_A$ (cf. equation \eqref{equation: div_zero_vector_fields_basis}):
                \begin{equation} \label{equation: explicit_commutators_between_central_basis_elements_and_derivations}
                    \begin{gathered}
                        \begin{cases}
                            [D_{r, s}, K_{a, b}]_{\extendedtoroidal_A} =
                            \begin{aligned}
                                & \left( (b + s_0)r - s(a + r_0) \right) K_{a - r + r_0, b - s + s_0}
                                \\
                                & + \delta_{(r - r_0, s - s_0), (a, b)} \left( r c_v + s c_t \right)
                            \end{aligned}
                            \\
                            [D_v, K_{a, b}]_{\extendedtoroidal_A} = a K_{a + r_0, b + s_0}
                            \\
                            [D_t, K_{a, b}]_{\extendedtoroidal_A} = b K_{a + r_0, b + s_0}
                        \end{cases}
                        \quad, \quad (a, b) \in \Z^2
                        \\
                        [D, c_v]_{\extendedtoroidal_A} = [D, c_t]_{\extendedtoroidal_A} = 0
                    \end{gathered}
                \end{equation}
        \end{lemma}
            \begin{proof}
                See \cite[Lemma 3.3.1.1]{msc_thesis_gamma_extended_toroidal_lie_algebras}.
            \end{proof}

    \subsection{Lie bialgebra structures on (extended) toroidal Lie algebras} \label{subsection: toroidal_lie_bialgebras}
        The central that we are trying to answer is the following.
        \begin{question} \label{question: classical_limit_of_affine_yangians}
            What is the classical limit of the coproduct structures on the Yangian of the untwisted affine Kac-Moody algebra $\hat{\g}$ that were constructed in \cite{guay_nakajima_wendlandt_affine_yangian_coproduct} and in \cite{ueda_coproduct_for_the_yangian_of_twisted_affine_type_A2}?
        \end{question}
        Issues of when such coproduct structures are defined aside (there are certain complications in type $\sfA_1^{(1)}$) and when the affine Yangian is even a flat deformation to begin with (so far, this is known to be true only when $\hat{\g}$ is simply laced), so that we can be assured that inquiring about the classical limit is even a well-posed question to begin with, we ought to begin answering question \ref{question: classical_limit_of_affine_yangians} by constructing some Lie bialgebra structure which can be nominated as a candidate for being the classical limit of the affine Yangian.
        
        Constructing Lie bialgebras begins with constructing Manin triples, and let us inspiration from the construction of the finite-type Yangian associated to $\g$. Recall that this is the\footnote{The uniqueness of the Yangian has recently been confirmed in \cite{gautam_wendlandt_xu_yangian_uniqueness}.} quantisation of a certain Lie bialgebra structure on the current algebra $\g[t] := \g \tensor \bbC[t]$. This Lie bialgebra structure comes from the $\Z$-graded Manin triple:
            \begin{equation} \label{equation: finite_yangian_manin_triples}
                \left( \g[t^{\pm 1}], \g[t], t^{-1}\g[t^{-1}] \right)
            \end{equation}
        wherein:
        \begin{itemize}
            \item the Lie algebra structure on $\g[t^{\pm 1}] := \g \tensor \bbC[t^{\pm 1}]$ is given as in equation \eqref{equation: current_algebra_brackets}, and $\g[t]$ and $t^{-1}\g[t^{-1}] := \g \tensor t^{-1}\bbC[t^{-1}]$ are Lie subalgebras of $\g[t^{\pm 1}]$ in the canonical manner,
            \item the symmetric, non-degenerate, and invariant bilinear form $(\cdot, \cdot)_{\g[t^{\pm}]}$ on $\g[t^{\pm}]$ is given as in equation \eqref{equation: current_algebra_pairing}, with $Z := \Z$ and $n_0 := -1$, and one checks that the Lie subalgebras $\g[t]$ and $t^{-1}\g[t^{-1}]$ are (maximally) isotropic with respect to this bilinear form,
            \item $\g[t]$ is regarded as a $\Z$-graded Lie algebra by means of:
                $$\deg(x f) := \deg f \quad, \quad \text{$f \in \bbC[t]$ homogeneous}$$
            and then the Lie algebra $t^{-1}\g[t^{-1}]$ is regarded as the $\Z$-graded dual of $\g[t]$, and we note that $\g[t^{\pm 1}] = \g[t] \oplus t^{-1}\g[t^{-1}]$.
        \end{itemize}
        The Manin triple \eqref{equation: finite_yangian_manin_triples} then gives rise to a quasi-triangular $\Z$-graded Lie bialgebra structure:
            \begin{equation}
                \dot{\delta}_{\rational}^+: \g[t] \to \g[t_1] \tensor \g[t_2^{-1}] 
            \end{equation}
        given by:
            \begin{equation} \label{equation: finite_yangian_cobracket}
                \begin{gathered}
                    \dot{\delta}_{\rational}^+(X) := \left[\Box(X), \dot{\calr}_{\rational} \right] \quad, \quad X \in \g[t]
                    \\
                    \dot{\calr}_{\rational} := \calr \cdot t_2^{-1} \1(t_2/t_1)^+ = \frac{\calr}{t_2 - t_1}
                \end{gathered}
            \end{equation}
        wherein:
            $$\calr := \calr_{\g, (\g^*)^{\op}}$$
        is the Casimir tensor\footnote{... or in other words, the canonical tensor of the non-degenerate bilinear form $(\cdot, \cdot)_{\Dr(\g)}$ that helps defines the standard Manin triple $( \Dr(\g), \g, (\g^*)^{\op} )$.} of $\g$, and $\1(t_2/t_1)^+$ is as in equation \eqref{equation: positive_negative_dirac_delta_distributions}.
        
        \begin{remark}
            \begin{itemize}
                \item We are to interpret $\dot{\delta}_{\rational}^+(X)$ as in equation \eqref{equation: finite_yangian_cobracket} as a formal sum of its components:
                    $$\dot{\delta}_{\rational}^+(X)_m := [\Box(X), \calr \cdot t_1^m t_2^{-m - 1}] \in t_1^m \g[t_1] \tensor t_2^{-m - 1} \g[t_2^{-1}] \quad, \quad \text{$X \in \g[t]$ homogeneous}$$
                \item One can also interpret the cobracket \eqref{equation: finite_yangian_cobracket} as a topological Lie bialgebra structure:
                    $$\dot{\delta}_{\rational}^+: \g[t] \to \g[t_1] \hattensor \g[\![t_2^{-1}]\!]$$
                wherein we regard the RHS as the $(t_1, t_2^{-1})$-adic closure of $\g[t_1] \tensor \g[t_2^{-1}]$ inside $(\g \tensor \g)[\![t_1, t_2^{-1}]\!]$.
            \end{itemize}
        \end{remark}

        \begin{remark}[Classical limit of finite-type dual Yangians] \label{remark: finite_dual_yangian_cobracket} 
            From equation \eqref{equation: positive_negative_dirac_delta_distributions}, we know that:
                $$\1(t_2/t_1)^- = -\1(t_2/t_1)^+$$
            At the same time, we know that the ($\Z$-graded) dual cobracket:
                $$\dot{\delta}_{\rational}^-: t^{-1}\g[t^{-1}] \to t_1^{-1}\g[t_1^{-1}] \tensor t_2\g[t_2]$$
            on $t^{-1}\g[t^{-1}] \cong \g[t]^{\star}$ is supposed to be given by:
                $$\dot{\delta}_{\rational}^- = ( \dot{\delta}_{\rational}^+ )^{\cop}$$
            Therefore, the dual cobracket can be given more succinctly by the following formula:
                \begin{equation} \label{equation: finite_dual_yangian_cobracket}
                    \begin{aligned}
                        \dot{\delta}_{\rational}^-(Y) & = [\Box(Y), -\dot{\calr}_{\rational}]
                        \\
                        & = [\Box(Y), \calr \cdot t_2^{-1} \1(t_2/t_1)^-]
                    \end{aligned}
                    \quad, \quad Y \in t^{-1}\g[t^{-1}]
                \end{equation}
        \end{remark}
        
        We shall now try to adapt the constrction of the cobracket \eqref{equation: finite_yangian_cobracket} in order to construct a $\Z^2$-graded Lie bialgebra structure on $\toroidal_A$, though first of all on $\extendedtoroidal_A$ (the former will be demonstrated to be a Lie sub-bialgebra of the latter). To this end, suppose henceforth that:
            $$(r_0, s_0) = (0, s_0) \not = (0, 0)$$
        and then let:
            \begin{equation} \label{equation: positive_and_negative_toroidal_lie_algebras}
                \begin{gathered}
                    \toroidal_A^+ := \uce( t^{-s_0 - 1} \g[v^{\pm 1}, t] ) \quad, \quad \toroidal_A^- := \uce( t^{s_0} \g[v^{\pm 1}, t^{-1}] )
                    \\
                    \z_A^+ := \z( \toroidal_A^+ ) \quad, \quad \z_A^- := \z( \toroidal_A^- )
                \end{gathered}
            \end{equation}
        Additionally, let:
            $$\d_A^{\pm} := ( \z_A^{\pm} )^{\star} \subset \d_A$$
        wherein $(-)^{\star}$ denotes graded duality. Finally, let:
            $$\extendedtoroidal_A^{\pm} := \toroidal_A^{\pm} \rtimes \d_A^{\pm}$$
        
        Via the fact that:
            \begin{equation} \label{equation: positive_and_negative_double_loop_algebras}
                t^{-s_0 - 1} \g[v^{\pm}, t] \cong \bigoplus_{(r, s) \in \Z \x \Z_{\geq 0}} \g \tensor \bbC v^r t^{s - s_0 - 1}
                \quad, \quad
                t^{s_0} \g[v^{\pm}, t^{-1}] \cong \bigoplus_{(r, s) \in \Z \x \Z_{\geq 0}} \g \tensor \bbC v^{-r} t^{-s + s_0}
            \end{equation}
        we have the following descriptions for $\z_A^{\pm}$:
            \begin{equation} \label{equation: positive_and_negative_toroidal_centres}
                \z_A^+ \cong \bigoplus_{(r, s) \in \Z \x \Z_{\geq 0}} \bbC K_{r, s} \oplus \bbC c_v
                \quad, \quad
                \z_A^- \cong \bigoplus_{(r, s) \in \Z \x \Z_{\geq 0}} \bbC K_{-r, -s + s_0} \oplus \bbC c_t
            \end{equation}
        with $K_{r, s}, c_v, c_t$ as in equation \eqref{equation: toroidal_centre_basis}. In turn, these explicit realisations for $\z_A^{\pm}$ tell us that:
            \begin{equation} \label{equation: positive_and_negative_div_zero_vector_fields}
                \d_A^- \cong \bigoplus_{(r, s) \in \Z \x \Z_{\geq 0}} \bbC D_{r, s} \oplus \bbC D_v
                \quad, \quad
                \d_A^+ \cong \bigoplus_{(r, s) \in \Z \x \Z_{< 0}} \bbC D_{r, s} \oplus \bbC D_t
            \end{equation}
        with $D_{r, s}, D_v, D_t$ as in equation \eqref{equation: div_zero_vector_fields_basis}.
                
        \begin{proposition}[Extended toroidal Manin triples] \label{prop: extended_toroidal_manin_triples}
            The triple of Lie algebras:
                \begin{equation} \label{equation: extended_toroidal_manin_triples}
                    \left( \extendedtoroidal_A, \extendedtoroidal_A^+, \extendedtoroidal_A^- \right)
                \end{equation}
            is a $\Z^2$-graded Manin triple if and only if:
                $$(r_0, s_0) = (0, -1)$$
        \end{proposition}
            \begin{proof}
                Through equation \eqref{equation: positive_and_negative_double_loop_algebras}, we see that:
                    $$( t^{-s_0 - 1} \g[v^{\pm}, t] )^{\star} \cong t^{s_0} \g[v^{\pm}, t]$$
                with respect to the non-degenerate bilinear form $(\cdot, \cdot)_{\extendedtoroidal_A}$ of degree $(r_0, s_0)$ if and only if $(r_0, s_0) = (0, -1)$, since the graded duality above holds if and only if:
                    $$(r_0, s_0) = (r, s - s_0 - 1) + (-r, -s + s_0)$$
                and after simplifying, this yields:
                    $$(r_0, s_0) = (0, -1)$$
            
                The rest follows automatically from the construction of the bilinear form $(\cdot, \cdot)_{\extendedtoroidal_A}$ (as in corollary \ref{coro: extended_toroidal_bilinear_forms}) and of the triple \eqref{equation: extended_toroidal_manin_triples}. For clarity, though, let us point out, in particular, that with the Lie algebras $\extendedtoroidal_A^{\pm}$, we have:
                    $$( \z_A^{\pm}, \d_A^{\pm} )_{\extendedtoroidal_A} = 0$$
                but at the same time, we have:
                    $$(\z_A^{\pm})^{\star} \cong \d_A^{\mp}$$
                via the non-degenerate bilinear form $(\cdot, \cdot)_{\extendedtoroidal_A}$, with both being consequences of the fact that $\deg D_{r, s} = (-r + r_0, -s + s_0)$ and $\deg D_v = \deg D_t = (r_0, s_0)$ (see corollary \ref{coro: extended_toroidal_bilinear_forms}); therefore, $\extendedtoroidal_A^{\pm}$ are isotropic when regarded as subspaces of $\extendedtoroidal_A$.
            \end{proof}
        \begin{remark}
            Primarily, we are interested in the case when the bilinear form $(\cdot, \cdot)_{\extendedtoroidal_A}$ has degree:
                $$(r_0, s_0) = (0, -1)$$
            because $\toroidal_A$ would then be classical limit of the Yangian of the affine Kac-Moody algebra $\hat{\g}$. Obviously, there are no differences beyond visual ones between the cases when $(r_0, s_0) = (0, -1) \not = (0, 0)$ and when $(r_0, s_0) = (-1, 0) \not = (0, 0)$, so it is not necessary to consider the latter explicitly. 

            It would also be interesting to consider cases when $r_0 = s_0 = -1$ simultaneously. In such cases, there will be two distinct families of toroidal Lie (bi)algebras, namely of the forms:
                $$\uce( \g[v, t] ) \quad, \quad \uce( v^{-1} \g[v^{-1}, t] )$$
            which when quantised, will yield us two families of quantum groups, with the first being \say{Yangian-like} in both variables, while the second is \say{dual-Yangian-like} in the first variables while \say{Yangian-like} in the second variable. To our knowledge, such quantisations have never before been considered in the literature, but we will defer such considerations until later, for doing so would deviate too far from the scope of the current work.
        \end{remark}

        The $\Z^2$-graded Manin triple from proposition \ref{prop: extended_toroidal_manin_triples}, like all graded Manin triples, defines a $\Z^2$-graded Lie bialgebra structure:
            $$\ddot{\delta}_{\rational}^+: \extendedtoroidal_A^+ \to \extendedtoroidal_A^+ \tensor \extendedtoroidal_A^+$$
        by means of the equation:
            $$( \ddot{\delta}_{\rational}^+(X), Y \tensor Z )_{\extendedtoroidal_A \tensor \extendedtoroidal_A} = ( X, [Y, Z]_{\extendedtoroidal_A^-} )_{\extendedtoroidal_A} \quad, \quad X \in \extendedtoroidal_A^+, Y, Z \in \extendedtoroidal_A^-$$
        (cf. equation \eqref{equation: lie_cobrackets_by_duality}). In principle, explicit expressions for the cobrackets:
            $$\ddot{\delta}_{\rational}^+(X) \quad, \quad X \in \extendedtoroidal_A^+$$
        can be obtained, because through equations \eqref{equation: toroidal_lie_algebra_commutators}, \eqref{equation: div_zero_vector_fields_commutators}, and \eqref{equation: explicit_commutators_between_central_basis_elements_and_derivations}, we now know all of the commutation relations between the basis elements of $\extendedtoroidal_A$. However, the necessary computations would be prohibitively cumbersome, so let us pursue the following strategy instead.

        Let:
            $$\ddot{\calr}_{\rational} \in \extendedtoroidal_A^+ \tensor \extendedtoroidal_A^-$$
        denote the canonical $2$-tensor associated to the non-degenerate bilinear form $(\cdot, \cdot)_{\extendedtoroidal_A}$. We shall attempt to demonstrate that:
            $$\ddot{\delta}_{\rational}^+(X) = [ \Box(X), \ddot{\calr}_{\rational} ] \quad, \quad X \in \extendedtoroidal_A^+$$
        thereby proving not only that $\ddot{\delta}_{\rational}^+: \extendedtoroidal_A^+ \to \extendedtoroidal_A^+ \tensor \extendedtoroidal_A^+$ is a $\Z^2$-graded Lie bialgebra structure, but moreover, that it is coboundary. To do so, we shall have to demonstrate that $\CYBE(\ddot{\calr}_{\rational}) \in \left( \bigwedge^3 \extendedtoroidal_A \right)^{\extendedtoroidal_A}$; in fact, it will even be shown that $\CYBE(\ddot{\calr}_{\rational}) = 0$, thus proving that $\ddot{\delta}_{\rational}^+$ is quasi-triangular. In any event, however, the first step shall be to write down an explicit description of $\ddot{\calr}_{\rational}$. \textit{A priori}, this can be done by first choosing a basis:
            $$\{ J_{r, s} \}_{(r, s) \in \Z \x \Z_{\geq 0}} \subset \extendedtoroidal_A^+ \quad, \quad \deg J_{r, s} = (r, s)$$
        for the $\Z \x \Z_{\geq 0}$-graded Lie algebra $\extendedtoroidal_A^+$, and then a graded-dual basis:
            $$\{ J_{r, s}^{\star} \}_{(r, s) \in \Z \x \Z_{\geq 0}} \subset \extendedtoroidal_A^- \quad, \quad \deg J_{r, s}^{\star} = (-r + r_0, -s + s_0)$$
        (with respect to the non-degenerate bilinear form $(\cdot, \cdot)_{\extendedtoroidal_A}$), and then we shall get:
            \begin{equation} \label{equation: extended_toroidal_classical_r_matrices_general_form}
                \ddot{\calr}_{\rational} = \sum_{(r, s) \in \Z \x \Z_{\geq 0}} J_{r, s} \tensor J_{r, s}^{\star}
            \end{equation}
        Now, the manner in which we constructed the Manin triple \eqref{equation: extended_toroidal_manin_triples} has already supplied us with such dual bases, namely:
            $$
                \begin{gathered}
                    \{ J_{r, s} \}_{(r, s) \in \Z \x \Z_{\geq 0}} := 
                    \begin{aligned}
                        & \{ x_i v^r t^s \}_{1 \leq i \leq \dim \g, (r, s) \in \Z \x \Z_{\geq 0}}
                        \\
                        & \cup \left( \{ K_{r, s} \}_{(r, s) \in \Z \x \Z_{\geq 0}} \cup \{ c_v \} \right)
                        \\
                        & \cup \left( \{ D_{-r + r_0, -s + s_0} \}_{(r, s) \in \Z \x \Z_{\geq 0}} \cup \{ D_t \} \right)
                    \end{aligned}
                    \\
                    \{ J_{r, s}^{\star} \}_{(r, s) \in \Z \x \Z_{\geq 0}} := 
                    \begin{aligned}
                        & \{ (x_i v^r t^s)^{\star} = x_i^* v^{-r + r_0} t^{-s + s_0} \}_{1 \leq i \leq \dim \g, (r, s) \in \Z \x \Z_{\geq 0}}
                        \\
                        & \cup \left( \{ K_{r, s}^{\star} = D_{r, s} \}_{(r, s) \in \Z \x \Z_{\geq 0}} \cup \{ c_v^{\star} = D_v \} \right)
                        \\
                        & \cup \left( \{ D_{-r + r_0, -s + s_0}^{\star} = K_{r - r_0, s - s_0} \}_{(r, s) \in \Z \x \Z_{\geq 0}} \cup \{ D_t^{\star} = c_t \} \right)
                    \end{aligned}
                \end{gathered}
            $$
        for some bases $\{x_i\}_{1 \leq i \leq \dim \g} \subset \g$ and $\{x_i^*\}_{1 \leq i \leq \dim \g} \subset \g^*$ that are dual to one another with respect to the non-degenerate bilinear form $(\cdot, \cdot)_{\g}$; note also, that because we are supposing that $(r_0, s_0) = (0, s_0) \not = (0, 0)$, the graded-dual basis $\{ J_{r, s}^* \}_{(r, s) \in \Z \x \Z_{\geq 0}}$ is indeed indexed by $\Z \x \Z_{< 0}$, and hence lies in the $\Z \x \Z_{< 0}$-graded Lie algebra $\extendedtoroidal_A^-$. Using these dual bases, we can then expand upon equation \eqref{equation: extended_toroidal_classical_r_matrices_general_form} as follows:
            $$
                \begin{aligned}
                    & \ddot{\calr}_{\rational}
                    \\
                    & = \sum_{(r, s) \in \Z \x \Z_{\geq 0}} J_{r, s} \tensor J_{r, s}^{\star}
                    \\
                    & =
                    \begin{aligned}
                        & \sum_{(r, s) \in \Z \x \Z_{\geq 0}} x_i v_1^r t_1^s \tensor (x_i v_2^r t_2^s)^{\star}
                        \\
                        & + \left( \sum_{(r, s) \in \Z \x \Z_{\geq 0}} K_{r, s} \tensor K_{r, s}^{\star} + c_{v_1} \tensor c_{v_2}^{\star} \right)
                        \\
                        & + \left( \sum_{(r, s) \in \Z \x \Z_{\geq 0}} D_{-r + r_0, -s + s_0} \tensor D_{-r + r_0, -s + s_0}^{\star} + D_{t_1} \tensor D_{t_2}^{\star} \right)
                    \end{aligned}
                    \\
                    & = 
                    \begin{aligned}
                        & \calr \cdot v_2^{r_0} \1(v_2/v_1) t_2^{s_0} \1(t_2/t_1)^+
                        \\
                        & + \left( \sum_{(r, s) \in \Z \x \Z_{\geq 0}} K_{r, s} \tensor D_{r, s} + c_{v_1} \tensor D_{v_2} \right)
                        \\
                        & + \left( \sum_{(r, s) \in \Z \x \Z_{\geq 0}} D_{-r + r_0, -s + s_0} \tensor K_{-r + r_0, -s + s_0} + D_{t_1} \tensor c_{t_2} \right)
                    \end{aligned}
                \end{aligned}
            $$
        \begin{theorem}[Extended toroidal Lie bialgebras] \label{theorem: extended_toroidal_lie_bialgebras}
            Suppose that:
                $$(r_0, s_0) = (0, -1)$$
            \begin{enumerate}
                \item The canonical tensor for the non-degenerate bilinear form $(\cdot, \cdot)_{\extendedtoroidal_A}$ is given explicitly by the following formula:
                    \begin{equation} \label{equation: extended_toroidal_classical_r_matrices}
                        \ddot{\calr}_{\rational} = (\calr + [...]) \cdot v_2^{r_0} \1(v_1 v_2^{-1}) t_2^{s_0} \1(t_1/t_2)^+ + \ddot{\calr}_v + \ddot{\calr}_t
                    \end{equation}
                \item Moreover, the $2$-tensor $\ddot{\calr}_{\rational} \in \extendedtoroidal_A \tensor \extendedtoroidal_A$ as in equation \eqref{equation: extended_toroidal_classical_r_matrices} above satisfies the CYBE \eqref{equation: CYBEs}, i.e. we have:
                    $$\CYBE( \ddot{\calr}_{\rational} ) = 0$$
                Thus, there is a quasi-triangular $\Z^2$-graded Lie bialgebra structure:
                    $$\ddot{\delta}_{\rational}^+: \extendedtoroidal_A^+ \to \extendedtoroidal_A^+ \tensor \extendedtoroidal_A^+$$
            \end{enumerate}
        \end{theorem}
            \begin{proof}
                \begin{enumerate}
                    \item Essentially, our job is to combine the $\bbC[v^{\pm 1}, t^{\pm 1}]$-coefficients in the expressions of the basis elements of $\g \tensor A, \z_A$, and $\d_A$ (see equations \ref{equation: toroidal_centre_basis} and \eqref{equation: div_zero_vector_fields_basis}).
                    \item 
                \end{enumerate}
            \end{proof}

    \section{Classical limits of affine Yangians}
    \subsection{Affine Yangians as flat deformations of toroidal Lie algebras}
        \todo[inline]{Definition of affine Yangians. Recall Curtis' proof from \cite{wendlandt_formal_shift_operators_on_yangian_doubles} that they are flat deformations of toroidal Lie algebras.}
\begin{definition} \cite{appel_gautam_wendlandt_R_matrices_of_affine_yangians}
    Let $\hbar \in\bbC^{\times} $. The Yangian $\rmY_{\hbar}(\g)$ is the unital, associative $\bbC$-algebra generated by elements ...
\end{definition}

        The assignment
            \begin{equation} \label{equation: peggy_will_name_this}
                \begin{gathered}
                    x_{i,r}^\pm \mapsto x_i^\pm  t_1^r
                    \\
                    x_{0,r}^\pm \mapsto x_{\mp\theta}  t_1^r t_2^{\pm 1}
                    \\
                    \xi_{i,r}\mapsto D_{i, i} h_i t_1^r
                    \\
                    \xi_{i,0} \mapsto h_\theta t_1^r + t_1^r t_2^{-1} dt_2
                \end{gathered}
            \end{equation}
        \todo[inline]{Dat: Should we use lowercase letters for classical generators and uppercase ones for quantum generators ?
        
        Peggy: Sure, or any other notation you prefer - I'm not picky and can stick with whatever makes sense for consistency. Also just want to be sure I have the right presentation on the toroidal side for when I start computations for real.
        
        Dat: Alright. I was just asking in case you already have a set of notations that works well, in which case I can adapt the toroidal Lie algebra notations accordingly. Also, I think it'll make our lives easier later on if we define the Yangian, say $\calY_{\hbar}(\hat{\g})$, formally over $\bbC[\![\hbar]\!]$ for the classical limit computation. One reason for this is that we want to establish a graded bialgebra isomorphism $\calY_{\hbar}(\hat{\g})/\hbar \cong \calU(\uce( \g[t_1^{\pm 1}, t_2] ))$, so that we can prove that $\calY_{\hbar}(\hat{\g})$ is a homogeneous quantisation of $\uce( \g[t_1^{\pm 1}, t_2] )$; this box is too small for details, but I mean the grading in which $\deg \hbar = 1$. If we work with a numerical value of $\hbar$ instead, then we will have to compute the associated graded algebra with respect to the canonical filtration on the Yangian, which I think will be more difficult.}
        
    extends to an epimorphism of $\bbC[\hbar]$-algebras. This leads to the classical limit  $\rmY_{\hbar}(\g) / \hbar \rmY_{\hbar}(\g) \cong \uce(\g \tensor A)....$

    
    \subsection{Coproducts on affine Yangians}
        \todo[inline]{For type $\sfA_2^{(2)}$, see Mamoru's paper \url{https://www.kurims.kyoto-u.ac.jp/~kyodo/kokyuroku/contents/pdf/2161-17.pdf}. I still don't know if we have a coproduct construction for type $\sfA_1^{(1)}$.}
    
    \addcontentsline{toc}{section}{References}
    \printbibliography

\end{document}