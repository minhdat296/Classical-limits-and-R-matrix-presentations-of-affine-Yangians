\section{Toroidal Lie (bi)algebras}
    \subsection{(Extended) toroidal Lie algebras}
        Let us begin by recalling the following description of universal central extensions (UCEs) of current algebras, i.e. of Lie algebras of the form $\g \tensor A$ wherein $A$ is a commutative algebra, due to Kassel. For a self-contained but comprehensive account of the following facts, we direct the reader to \cite[Section 2.2]{msc_thesis_gamma_extended_toroidal_lie_algebras}.
        
        First of all, for brevity, let us write $xf$ to denote the pure tensors $x \tensor f \in \g \tensor A$. On $\g \tensor A$, the Lie algebra structure is given by:
            $$[xf, yg]_{\g \tensor A} := [x, y]_{\g} fg$$
        It is worth noting, that with this Lie bracket, $\g \tensor A$ is a perfect Lie algebra\footnote{A Lie algebra is said to be perfect if it is equal to its derived subalgebra.} as a consequence of the simplicity of the Lie algebra $\g$ that we assumed from before. This fact is important because Lie algebras are perfect if and only if they admit UCEs (see \cite[Lemma 1.10]{garland_arithmetics_of_loop_groups}); for a perfect Lie algebra $\a$, we denote its UCE by $\uce(\a)$. It is also desirable to have a description of the structure of $\toroidal_A$, and since:
            $$\toroidal_A := \uce(\g \tensor A) \cong ( \g \tensor A ) \oplus^{\e} \z_A$$
        wherein $\e: \bigwedge^2 (\g \tensor A) \to \z_A$ is the universal Lie $2$-cocycle with values in the centre $\z_A := \z( \toroidal_A )$, it suffices to give a more detailed description of the centre. In \cite{kassel_universal_central_extensions_of_lie_algebras}, Kassel demonstrated via the use of cyclic cohomology, that the centre can be given via:
            $$\z_A \cong \bar{\Omega}^1_A := \Omega^1_A/dA \quad, \quad \e(xf, yg) := (x, y)_{\g} g \bar{d}f$$
        wherein $\Omega^1_A$ denotes the $A$-module of algebraic $1$-forms on $A$, regarded here as a $\bbC$-vector space, $d: A \to \Omega^1_A$ is the initial object in the category of $A$-module homomorphisms $d': A \to \Omega'$ such that $d'(fg) = fdg + gdf$ (see \cite[\href{https://stacks.math.columbia.edu/tag/00RM}{Tag 00RM}]{stacks-project} for more details), and we have written $\bar{d}f$ for the image of $df \in \Omega^1_A$ under the canonical quotient map $\Omega^1_A \to \bar{\Omega}^1_A$.

        More can be said when $A$ is graded by some abelian group $Z$, say $A := \bigoplus_{n \in Z} A_n$. In this case, $\g \tensor A$ and $\Omega^1_A$ carry an induced $Z$-grading given by:
            $$\g \tensor A = \bigoplus_{n \in Z} (\g \tensor A)_n \quad, \quad (\g \tensor A)_n := \g \tensor A_n$$
            $$\Omega^1_A = \bigoplus_{n \in Z} (\Omega^1_A)_n \quad, \quad (\Omega^1_A)_n := \bigoplus_{ \substack{i, j \in Z\\i + j = n} } \frac{A_i \tensor A_j}{}$$
        Together, these two induced $Z$-gradings induce a further $Z$-grading on $\toroidal_A$, determined by:
            $$\bar{\Omega}^1_A := \bigoplus_{n \in Z} ( \bar{\Omega}^1_A )_n, \quad, \quad ( \bar{\Omega}^1_A )_n := ( \Omega^1_A )_n/dA_n$$
        At the same time, for a distinguished element $n_0 \in Z$, one can define a symmetric, non-degenerate, and invariant bilinear form on $\g \tensor A$ given by:
            $$(x f, y g)_{\g \tensor A} := (x, y)_{\g} \delta_{ \deg f + \deg g, n_0 }$$
        This form extends naturally to $\toroidal_A$, though by invariance, this extension does not retain the non-degeneracy property: in fact, it is easy to show that the radical of this extended bilinear form - which shall be denoted by $(\cdot, \cdot)_{\toroidal_A}$ - is precisely the centre $\z_A$.

        The fact that the bilinear form $(\cdot, \cdot)_{\toroidal_A}$ is no longer non-degenerate prompts us to enlarge $\toroidal_A$ by a vector space $\divzero_A$, i.e. we form:
            $$\extendedtoroidal_A := \toroidal_A \oplus \divzero_A$$
        in such a way that the extension of the bilinear form $(\cdot, \cdot)_{\toroidal_A}$ to $\extendedtoroidal_A$, which shall be denoted by:
            $$(\cdot, \cdot)_{\extendedtoroidal_A}$$
        becomes non-degenerate. This means that the bilinear form $(\cdot, \cdot)_{\extendedtoroidal_A}$ is to pair $\divzero_A$ non-degenerately with $\z_A$. In \cite{msc_thesis_gamma_extended_toroidal_lie_algebras}, it was demonstrated that this data is enough to specify a family of Lie algebra structures on $\extendedtoroidal_A$, parametrised by (abelian) Lie $2$-cocycles $\sigma \in H^2_{\Lie}(\divzero_A, \z_A)$. We care in particular about the cocycle $0$, corresponding to the semi-direct product:
            $$\toroidal_A \rtimes \divzero_A$$
        and henceforth, this shall be the Lie algebra structure that we shall equip $\extendedtoroidal_A$ with. Now, we should also mention that, due to the fact that $\z_A \cong \bar{\Omega}^1_A$, one can identify $\divzero_A$ with a Lie subalgebra of the Lie algebra $\der(A)$ of derivations on $A$; it is also possible to compute an explicit basis for $\divzero_A$. Which Lie subalgebra $\d_A$ is isomorphic to in particular, though, depends heavily on the degree of $(\cdot, \cdot)_{\extendedtoroidal_A}$. Consequently, how the Lie brackets between basis elements of $\extendedtoroidal$ are given also depends on the degree of this bilinear form.

        For us, the cases of particular interest are the following. From now on, let $A := \bbC[t_1^{\pm 1}, t_2^{\pm 1}]$, let the grading abelian group be $Z := \Z^2$, and the cases that we shall consider are when the distinguished element $n_0 \in \Z^2$ are $n_0 = (0, -1)$ and $(0, 0)$. The reason for this is that in these cases, $\toroidal_A$ is to be the classical limits of - respectively - the centrally extended double Yangian associated to the affine Kac-Moody algebra $\hat{\g}$ associated to $\g$ (i.e. the Lie affinisation of $\g$) and the quantum toroidal algebra associated to $\g$ (i.e. the quantum affinisation associated to $\hat{\g}$). We expect the latter to somehow \say{degenerate} to the former, so it is important to analyse the relationship between their classical limits as well.

        Before we can do that, though, we must endow $\toroidal_A$ with a Lie bialgebra structure. There is, however, an obstacle: the theory of Manin triples tells us that it is necessary to have a non-degenerate, symmetric, and invariant bilinear form, yet we know that $(\cdot, \cdot)_{\toroidal_A}$ is \textit{degenerate}. That said, we do have a non-degenerate, symmetric, and invariant bilinear form on the larger Lie algebra $\extendedtoroidal_A$, namely $(\cdot, \cdot)_{\extendedtoroidal_A}$, meaning that we can construct a Lie bialgebra structure on $\extendedtoroidal_A$ instead. We shall then show in subsection \ref{subsection: toroidal_lie_bialgebras} that $\toroidal_A$ is a Lie sub-bialgebra of $\extendedtoroidal_A$.

        Let us now recall the explicit structures of $\extendedtoroidal_A$ in the two cases mentioned above. Our main reference is \cite{msc_thesis_gamma_extended_toroidal_lie_algebras}.

    \subsection{Lie bialgebra structures} \label{subsection: toroidal_lie_bialgebras}