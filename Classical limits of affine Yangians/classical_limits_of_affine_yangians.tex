\section{Classical limits of affine Yangians}
    \subsection{Affine Yangians as flat deformations of toroidal Lie algebras}
        \todo[inline]{Definition of affine Yangians. Recall Curtis' proof from \cite{wendlandt_formal_shift_operators_on_yangian_doubles} that they are flat deformations of toroidal Lie algebras.}
\begin{definition} \cite{appel_gautam_wendlandt_R_matrices_of_affine_yangians}
    Let $\hbar \in\bbC^{\times} $. The Yangian $\rmY_{\hbar}(\g)$ is the unital, associative $\bbC$-algebra generated by elements ...
\end{definition}

        The assignment
            \begin{equation} \label{equation: peggy_will_name_this}
                \begin{gathered}
                    x_{i,r}^\pm \mapsto x_i^\pm  t_1^r
                    \\
                    x_{0,r}^\pm \mapsto x_{\mp\theta}  t_1^r t_2^{\pm 1}
                    \\
                    \xi_{i,r}\mapsto D_{i, i} h_i t_1^r
                    \\
                    \xi_{i,0} \mapsto h_\theta t_1^r + t_1^r t_2^{-1} dt_2
                \end{gathered}
            \end{equation}
        \todo[inline]{Dat: Should we use lowercase letters for classical generators and uppercase ones for quantum generators ?
        
        Peggy: Sure, or any other notation you prefer - I'm not picky and can stick with whatever makes sense for consistency. Also just want to be sure I have the right presentation on the toroidal side for when I start computations for real.
        
        Dat: Alright. I was just asking in case you already have a set of notations that works well, in which case I can adapt the toroidal Lie algebra notations accordingly. Also, I think it'll make our lives easier later on if we define the Yangian, say $\calY_{\hbar}(\hat{\g})$, formally over $\bbC[\![\hbar]\!]$ for the classical limit computation. One reason for this is that we want to establish a graded bialgebra isomorphism $\calY_{\hbar}(\hat{\g})/\hbar \cong \calU(\uce( \g[t_1^{\pm 1}, t_2] ))$, so that we can prove that $\calY_{\hbar}(\hat{\g})$ is a homogeneous quantisation of $\uce( \g[t_1^{\pm 1}, t_2] )$; this box is too small for details, but I mean the grading in which $\deg \hbar = 1$. If we work with a numerical value of $\hbar$ instead, then we will have to compute the associated graded algebra with respect to the canonical filtration on the Yangian, which I think will be more difficult.}
        
    extends to an epimorphism of $\bbC[\hbar]$-algebras. This leads to the classical limit  $\rmY_{\hbar}(\g) / \hbar \rmY_{\hbar}(\g) \cong \uce(\g \tensor A)....$

    
    \subsection{Coproducts on affine Yangians}
        \todo[inline]{For type $\sfA_2^{(2)}$, see Mamoru's paper \url{https://www.kurims.kyoto-u.ac.jp/~kyodo/kokyuroku/contents/pdf/2161-17.pdf}. I still don't know if we have a coproduct construction for type $\sfA_1^{(1)}$.}