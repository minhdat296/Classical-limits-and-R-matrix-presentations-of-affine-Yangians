\section{Affine (double) Yangians as associative algebras}
    \subsection{Affine Yangians}
        \todo[inline]{Dat: The definitions in Curtis' papers work as long as $\hat{\g}$ isn't of type $\sfA_1^{(1)}$. For the $\sfA_1^{(1)}$ case, see the papers \url{https://arxiv.org/abs/1404.5240}, \url{https://arxiv.org/abs/1605.01314}, and \url{https://arxiv.org/abs/1512.09109} of Bershtein and Tsymbaliuk.}
    
        \todo[inline]{Definition of affine Yangians of simply laced types. Recall Curtis' proof from \cite{wendlandt_formal_shift_operators_on_yangian_doubles} that they are flat deformations of toroidal Lie algebras of the corresponding types.}
        \begin{definition} \cite{appel_gautam_wendlandt_R_matrices_of_affine_yangians}
            Let $\hbar \in\bbC^{\times} $. The Yangian $\rmY_{\hbar}(\g)$ is the unital, associative $\bbC$-algebra generated by elements ...
        \end{definition}
        
                The assignment
                    \begin{equation} \label{equation: peggy_will_name_this}
                        \begin{gathered}
                            x_{i,r}^\pm \mapsto x_i^\pm  v^r
                            \\
                            x_{0,r}^\pm \mapsto x_{\mp\theta}  v^r t^{\pm 1}
                            \\
                            \xi_{i,r}\mapsto D_{i, i} h_i v^r
                            \\
                            \xi_{i,0} \mapsto h_\theta v^r + v^r t^{-1} dt
                        \end{gathered}
                    \end{equation}
                \todo[inline]{Dat: Should we use lowercase letters for classical generators and uppercase ones for quantum generators ?
                
                Peggy: Sure, or any other notation you prefer - I'm not picky and can stick with whatever makes sense for consistency. Also just want to be sure I have the right presentation on the toroidal side for when I start computations for real.
                
                Dat: Alright. I was just asking in case you already have a set of notations that works well, in which case I can adapt the toroidal Lie algebra notations accordingly. Also, I think it'll make our lives easier later on if we define the Yangian, say $\calY_{\hbar}(\hat{\g})$, formally over $\bbC[\![\hbar]\!]$ for the classical limit computation. One reason for this is that we want to establish a graded bialgebra isomorphism $\calY_{\hbar}(\hat{\g})/\hbar \cong \calU(\uce( \g[v^{\pm 1}, t] ))$, so that we can prove that $\calY_{\hbar}(\hat{\g})$ is a homogeneous quantisation of $\uce( \g[v^{\pm 1}, t] )$; this box is too small for details, but I mean the grading in which $\deg \hbar = 1$. If we work with a numerical value of $\hbar$ instead, then we will have to compute the associated graded algebra with respect to the canonical filtration on the Yangian, which I think will be more difficult.}
                
            extends to an epimorphism of $\bbC[\hbar]$-algebras. This leads to the classical limit  $\rmY_{\hbar}(\g) / \hbar \rmY_{\hbar}(\g) \cong \uce(\g \tensor A)....$

    \subsection{Affine double Yangians}

    \subsection{Affine (double) Yangians as flat deformations of toroidal Lie algebras} \label{subsection: affine_(double)_yangians_as_flat_deformations_of_toroidal_lie_algebras}
        \todo[inline]{Dat: This is the part that depends on $\hat{\g}$ being simply laced. At the moment, I don't know if affine (double) Yangians being flat deformations of toroidal Lie algebras - which is equivalent to PBW for $\calY_{\hbar}(\hat{\g})$ - is true when we relax the simply-laced-ness assumption, so when $\hat{\g}$ is \textit{not} simply laced, it does not make sense yet to think about the classical limit of $\calY_{\hbar}(\hat{\g})$. I don't see why PBW shouldn't be true for $\calY_{\hbar}(\hat{\g})$ in general, but all of the proofs I know of at the moment use vertex operator techniques, and the technical difficulty lies in the root lattice part of the Fock space representation.}