\section{Affine Yangians as associative algebras}
    \begin{convention} \label{conv: a_fixed_symmetrisable_kac_moody_algebra}
        Throughout, we fix a symmetrisable Kac-Moody algebra $\g$ over $\bbC$, though in principle, we can work over any algebraically closed field characteristic $0$. On this Lie algebra, we choose once and for all a symmetric, non-degenerate, and invariant bilinear form $(\cdot, \cdot)_{\bar{\g}}$; \textit{a priori}, any such bilinear form is a non-zero scalar multiple of a standard one (see \cite[Chapter 2]{kac_infinite_dimensional_lie_algebras} for more details).
        
        Let us fix also a Cartan subalgebra $\h \subset \g$ of dimension $l$. By means of the bilinear form fixed above, we can define a root system $\rootsystem \subset \h^*$, in which we choose a subset of simple roots indexed by a set $\simpleroots$ of cardinality $n \geq 1$. With respect to these choices, we have a set of Chevalley-Serre generators $\{h_i, e_i^{\pm}\}_{i \in \simpleroots}$, normalised so that $(h_i, h_j)_{\g} = \1_{i, j}$, wherein $\1_{i, j}$ is the Kronecker delta.
    \end{convention}

    Suppose that:
        $$\{\alpha_i\}_{1 \leq i \leq n}$$
    is an enumeration of the set of simple roots of $\g$. When $\g$ is of an affine type in Kac's classification from \cite[Chapter 4]{kac_infinite_dimensional_lie_algebras}, the set of simple roots can alternatively be realised as:
        $$\{\alpha_0\} \cup \{\alpha_i\}_{1 \leq i \leq l}$$
    (and so in particular, we have $n = l + 1$), with $l \geq 1$ as in convention \ref{conv: underlying_finite_type_kac_moody_algebra}. In this case, let us assume that the underlying finite-type Kac-Moody algebra of $\g$ is the finite-dimensional simple Lie algebra $\bar{\g}$ from convention \ref{conv: underlying_finite_type_kac_moody_algebra}.

    \begin{convention}
        Additionally, let us say \say{affine Lie algebra} to mean the derived subalgebra of a Kac-Moody algebra of affine type.
    \end{convention}

    For more details on Kac-Moody algebras, we refer the reader to subsection \ref{subsection: setup_kac_moody_algebras}.

    \subsection{Affine Yangians by Drinfeld currents}
        We take the following presentation from \cite[Proposition 2.3]{guay_regelskis_wendlandt_affine_yangian_vertex_representations_and_PBW} as our working definition of the Yangian associated to the derived subalgebra of a symmetrisable Kac-Moody algebra $\g$.
        \begin{definition}[Yangians of symmetrisable Kac-Moody algebras ($\g \not = \hat{\sl}_2$)] \label{def: yangians_of_symmetrisable_kac_moody_algebras_by_drinfeld_currents}
            The \textbf{Yangian} associated to the derived subalgebra $\g'$ of the symmetrisable Kac-Moody algebra $\g$ is the associative and unital algebra:
                $$\calY_{\hbar}(\g')$$
            generated by the coefficients of the following formal power series in $\calY_{\hbar}(\g')[\![z^{-1}]\!]$:
                \begin{equation} \label{equation: kac_moody_yangian_drinfeld_currents}
                    \begin{gathered}
                        X_i^{\pm}(z) := \sum_{r \geq 0} X_i^{\pm}[r] z^{-r - 1}
                        \\
                        H_i^{\pm}(z) := \sum_{r \geq 0} H_i^{\pm}[r] z^{-r - 1}
                    \end{gathered}
                    \quad, \quad i \in \simpleroots
                \end{equation}
            which are subjected to the following relations taking place in $\calY_{\hbar}[\![z^{-1}, w^{-1}]\!]$:
                \begin{equation} \label{equation: kac_moody_yangian_drinfeld_current_relations}
                    \begin{gathered}
                        H_i[0] = D_{i, i} h_i \quad, \quad [H_i(z), H_j(w)] = 0
                        \\
                        [H_i[0], X_j^{\pm}(z)] = \pm \alpha_j( h_i ) X_j^{\pm}(z)
                        \\
                        [...]
                    \end{gathered}
                    \quad, \quad i, j \in \simpleroots
                \end{equation}
        \end{definition}
        \begin{remark}[Why $\g'$ ?]
            Had we replaced $\g'$ in definition \ref{def: yangians_of_symmetrisable_kac_moody_algebras_by_drinfeld_currents} with the full Kac-Moody algebra $\g$, we would have obtained a similar algebra. For representation-theoretic purposes, these algebras are almost interchangeable, and arguments can even be made for the latter being better suited for such purposes. However, for structural inquiries, it would seem that $\calY_{\hbar}(\g')$ is more convenient to work with, primarily because we are permitted to think of this algebra as flat deformation of the universal enveloping algebra of $\uce( \g'[t] )$; note that this is nothing but $\bar{\g}[t]$ itself when $\g = \bar{\g}$ is a finite-type Kac-Moody algebra (in which case $\bar{\g}' = \bar{\g}$).
            
            In slightly more details, $\calY_{\hbar}(\g)$ is can be realised as the semi-direct product of $\calY_{\hbar}(\g')$ with the universal enveloping algebra of a rather small finite-dimensional abelian Lie algebra (see \cite[Proposition 3.2]{wendlandt_formal_shift_operators_on_yangian_doubles}). Moreover, while $\calY_{\hbar}(\g')$ deforms $\toroidal_A^+$ (see \cite[Theorem 6.9]{guay_regelskis_wendlandt_affine_yangian_vertex_representations_and_PBW}), the larger Yangian $\calY_{\hbar}(\g)$ instead deforms the semi-direct product of $\toroidal_A^+$ with the aforementioned finite-dimensional abelian Lie algebra (\cite[Proposition 3.7]{wendlandt_formal_shift_operators_on_yangian_doubles}). As an aside, we would like to mention that the analogues of these two results when $\hbar$ is specialised to some non-zero complex number are also true, up to a certain topological completion; see \cite[Subsection 6.1]{wendlandt_formal_shift_operators_on_yangian_doubles} for more details. Also, while these two results are at present only known to be true when $\g$ is either of a finite or a simply laced affine type, they are expected to hold for non-simply laced affine types as well\footnote{These difficulties are present because proving that a certain algebra is a flat deformation of a universal enveloping algebra is the same as proving that said algebra has admits PBW bases. One technique for doing so in the case of $\calY_{\hbar}(\g')$ is to construct a faithful representation of it on a Fock space of a Heisenberg algebra, which depends on where given the Kac-Moody algebra $\g$ falls into Kac's classification scheme. See \cite[Sections 5 and 6]{guay_regelskis_wendlandt_affine_yangian_vertex_representations_and_PBW} for more details.}.
            
            At the same time, let us note that we can realise $\toroidal_A^+$ as a Lie algebra extension of the current algebra $\g'[t] \cong (\bar{\g}[v^{\pm 1}] \oplus \bbC \level) \tensor \bbC[t] \cong \bar{\g}[v^{\pm 1}, t] \oplus \bbC[t]$. On the other hand, it is not clear if the semi-direct product of $\toroidal_A^+$ with the finite-dimensional abelian Lie algebra mentioned earlier would be an extension of the larger current algebra $\g[t]$. As the Yangian at play is supposed to deform the universal enveloping algebra of $\uce(\g'[t])$, we are of the opinion that it is more natural to work with $\calY_{\hbar}(\g')$. In fact, it been shown in \cite[Proposition 4.7]{guay_regelskis_wendlandt_affine_yangian_vertex_representations_and_PBW} that:
                $$\toroidal_A^+ \cong \uce( \g'[t] )$$
            whenever $\g$ is of an \textit{untwisted} affine type in Kac's classification from \cite[Chapter 4]{kac_infinite_dimensional_lie_algebras}.
        \end{remark}
        
        When it comes to defining Yangians associated to affine Lie algebras, the $\g = \hat{\sl}_2$ case is somewhat exceptional. The following is from \cite{tsymbaliuk_classical_limits_of_type_A_toroidal_QUEs_and_affine_yangians}.
        \begin{definition}[Yangian of $\hat{\sl}_2'$] \label{def: yangian_of_affine_sl_2_by_drinfeld_currents}
            
        \end{definition}

    \subsection{\textit{Intermission}: Current presentation for toroidal Lie algebras}
        Now, let $\g$ be an affine Kac-Moody algebra.
    
        Since the Yangian $\calY_{\hbar}(\g')$ of an an affine Lie algebra $\g'$ by means of Drinfeld currents, in order to verify that said Yangian deforms $\calU(\toroidal_A^+)$, we must first provide a realisation of the toroidal Lie algebra $\toroidal_A^+$ in terms of similar Drinfeld-style currents. As suggested in \cite[Section 4]{guay_regelskis_wendlandt_affine_yangian_vertex_representations_and_PBW}, such a presentation for $\toroidal_A^+$ can be obtained via an explicit isomorphism:
            $$\toroidal_A^+ \xrightarrow[]{\cong} \caly$$
        with $\caly := \uce(\g'[t])$, the notation supposing to remind us of the fact that it is $\uce(\g'[t])$ that ought to be thought of as \textit{the} classical limit of $\calY_{\hbar}(\g')$.

        \begin{definition}[Toroidal Lie algebra by Drinfeld currents] \label{def: toroidal_lie_algebras_by_drinfeld_currents}
            (Cf. \cite[Definition 2.5]{guay_regelskis_wendlandt_affine_yangian_vertex_representations_and_PBW}) We shall write:
                $$\caly^{\Dr}$$
            to mean the Lie algebra generated by the following formal power series in 
        \end{definition}

    \subsection{Affine Yangians as flat deformations of toroidal Lie algebras} \label{subsection: affine_yangians_as_flat_deformations_of_toroidal_lie_algebras}
        \todo[inline]{Dat: This is the part that depends on $\g$ being simply laced. At the moment, I don't know if affine Yangians being flat deformations of toroidal Lie algebras - which is equivalent to PBW for $\calY_{\hbar}(\g)$ - is true when we relax the simply-laced-ness assumption, so when $\g$ is \textit{not} simply laced, it does not make sense yet to think about the classical limit of $\calY_{\hbar}(\g)$. I don't see why PBW shouldn't be true for $\calY_{\hbar}(\g)$ in general, but all of the proofs I know of at the moment use vertex operator techniques, and the technical difficulty lies in the root lattice part of the Fock space representation.}