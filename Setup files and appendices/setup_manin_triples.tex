First, let us recall the definition of Lie bialgebras, as written down in \cite[Section 1]{etingof_kazhdan_quantisation_1} (originally from \cite{drinfeld_quantum_groups}).
\begin{definition}[Topological Lie coalgebras] \label{def: topological_lie_coalgebras}
    A \textbf{topological Lie coalgebra} is a pair:
        $$(\a, \delta)$$
    consisting of a topological vector space $\a$ along with a linear map:
        $$\delta: \a \to \a \hattensor \a$$
    usually called the \textbf{Lie cobracket}, such that its dual defines a Lie bracket $\delta^*: \a^* \tensor \a^* \to \a$.
\end{definition}
\begin{remark}
    Note that even though we ostensibly get a map $\delta^*: (\a^* \hattensor \a^*) \to \a^*$ by dualising $\delta$, we also have $\a^* \tensor \a^* \subset (\a \hattensor \a)^*$ always, so we can restrict the domain of the dual map in order to get a linear map $\delta^*: \a^* \tensor \a^* \to \a$. Afterwards, we check if this satisfies the axioms defining a Lie bracket.
\end{remark}
\begin{definition}[Topological Lie bialgebras] \label{def: topological_lie_bialgebras}
    A \textbf{topological Lie bialgebra} is a Lie coalgebra $(\a, \delta)$ wherein $\a$ is a Lie algebra (say with Lie bracket $[\cdot, \cdot]$), such that $\delta \in Z^1_{\Lie}(\a, \a \hattensor \a)$.
\end{definition}
\begin{remark}
    Concretely, what the latter condition in definition \ref{def: topological_lie_bialgebras} is trying to say, is that a Lie cobracket $\delta$ is a Lie bialgebra structure on a Lie algebra $(\a, [\cdot, \cdot])$ if and only if $\delta$ is a Lie derivation with respect to $[\cdot, \cdot]$. That is, the following version of the Leibniz rule is satisfied:
        \begin{equation} \label{equation: lie_bialgebra_cocycle_condition}
            \delta( [x, y] ) = [\delta(x), \Box(y)] + [\Box(x), \delta(y)] \quad, \quad x, y \in \a
        \end{equation}
    We remind the reader also, that we are regarding $\a \hattensor \a$ as an $\a$-module via the composite action:
        $$\calU(\a) \xrightarrow[]{\Box} \calU(\a)^{\tensor 2} \xrightarrow[]{\ad_{\a} \tensor \ad_{\a}} \gl(\a \hattensor \a)$$
    wherein $\Box(x) := x \tensor 1 + 1 \tensor x$ for all $x \in \a$ is the canonical coproduct on the universal enveloping algebra $\calU(\a)$.
\end{remark}
\begin{remark} \label{remark: poisson_lie_groups}
    \todo[inline]{Lie bialgebras as tangent spaces of Poisson-Lie groups. CYBE is satsified iff the Schouten bracket of the Poisson structure/$2$-vector field vanishes, as this is precisely what is needed to have integrability.}
\end{remark}

Next, let us recall how Lie bialgebra structures can be constructed by means of so-called \say{Manin triples}. We begin with the following definition, which is a slight modification of the definition from \cite[Subsection 2.6]{appel_laredo_2_categorical_etingof_kazhdan_quantisation}. See also \cite[Subsection 7.4]{etingof_kazhdan_quantisation_1} and \cite[Subsection 1.3.B]{chari_pressley_quantum_groups}, Definition 1.3.3 on p. 27 and the remark on p. 28 in particular.
\begin{definition}[Manin triples] \label{def: manin_triples}
    A \textbf{Manin triple} is a triple of Lie algebras:
        $$(\a, \a^+, \a^-)$$
    together with a non-degenerate invariant pairing:
        $$(\cdot, \cdot)_{\a} \in \Hom( \Sym^2(\a)^{\a}, \bbk )$$
    which are to satisfy the following conditions.
    \begin{itemize}
        \item $\a^{\pm}$ are Lie subalgebras of $\a$.
        \item $\a = \a^- \oplus \a^+$ as vector spaces (but not necessarily as Lie algebras).
        \item $\a^{\pm}$ are isotropic to one another with respect to $(\cdot, \cdot)_{\a}$, i.e. $(\a^{\pm}, \a^{\pm})_{\a} = 0$.
        \item We endow $\a^{\mp}$ with the discrete topology and their dual spaces $(\a^{\mp})^*$ with the weak topology, and then require that the linear maps:
            \begin{equation} \label{equation: manin_triple_currying}
                \beta^{\pm}: \a^{\pm} \xrightarrow[]{\cong} (\a^{\mp})^*
            \end{equation}
        given by:
            $$\beta^{\pm}(x) := (x, \cdot)_{\a} \quad, \quad x \in \a^{\pm}$$
        are isomorphisms of topological vector spaces\footnote{That is to say, they are continuous linear isomorphisms.}. Moreover, the Lie bracket $[\cdot, \cdot]_{\a}$ is to be continuous with respect to the product topology on $\a = \a^+ \oplus \a^-$.
        \item By dualising the Lie brackets $[\cdot, \cdot]_{\a^{\mp}}: \a^{\mp} \tensor \a^{\mp} \to \a^{\mp}$, one obtains linear maps:
            $$\delta^{\pm} := [\cdot, \cdot]_{\a^{\mp}}^*: (\a^{\mp})^* \to (\a^{\mp} \tensor \a^{\mp})^*$$
        whose codomain lies in the subspaces $(\a^{\mp})^* \hattensor (\a^{\mp})^* \subseteq (\a^{\mp} \tensor \a^{\mp})^*$.
    \end{itemize}
\end{definition}
\begin{remark}[Finite-dimensional Manin triples]
    When $\a$ is finite-dimensional (equivalently, when either of $\a^{\pm}$ are finite-dimensional), it is sufficient to only impose the first three conditions in definition \ref{def: manin_triples} in order to ensure that $(\a, \a^+, \a^-)$ is a Manin triple.
\end{remark}

\begin{remark}[Completed tensor products] \label{remark: completed_tensor_products}
    Before moving on, let us also make some comments on the topological aspects of definition \ref{def: manin_triples}, as well as when and how they can be safely done away with. In particular, we would like to see when it is possible to have linear maps:
        $$\delta^{\pm} := [\cdot, \cdot]_{\a^{\mp}}^*: (\a^{\mp})^* \to (\a^{\mp} \tensor \a^{\mp})^*$$
    whose codomains are contained inside the subspaces $(\a^{\mp})^* \hattensor (\a^{\mp})^* \subseteq (\a^{\mp} \tensor \a^{\mp})^*$.

    For us, any algebraic structure (e.g. algebras, modules, etc.) that is endowed with a topology, shall always be \say{adic}\footnote{These algebraic structures are otherwise referred to as being \say{linearly topologised}. For details on this notion, we refer the reader to \cite[\href{https://stacks.math.columbia.edu/tag/07E7}{Tag 07E7} and \href{https://stacks.math.columbia.edu/tag/0AMQ}{Tag 0AMQ}]{stacks-project}.}. We do \textit{not} assume that topologised algebraic structures are complete with respect to the topologies that they are endowed with, as there are many practical examples (e.g. loop algebras) which are topologically incomplete.
    \begin{enumerate}
        \item The first of our issues is that, as it stands, it is now clear what we mean when we write down \say{the completed tensor product} $\hattensor$. Luckily, for adic modules\footnote{... and hence, algebras also, which we regard as monoid objects in categories of modules.}:
            $$M := \projlim_{i \in \calI} M_i \quad, \quad N := \projlim_{j \in \calJ} N_j$$
        over an adic ring $R$, their algebraic tensor product $M \tensor_R N$ is naturally endowed with the linear topology in which the basic open subsets are given by:
            $$(M \tensor N)_{i, j} := \im\left( M_i \tensor_R N + M \tensor_R N_j \to M \tensor_R N \right)$$
        With respect to this topology, \textit{the} completed tensor product of $M$ and $N$ is given by:
            $$M \hattensor_R N := \projlim_{(i, j) \in \calI \x \calJ} (M \tensor_R N)_{i, j}$$
        \item In cases wherein $R$ is a valuation ring (e.g. $R := \bbC[\![\hbar]\!]$)
        
        \todo[inline]{Completed tensor products over local fields are "generic fibres" of completed tensor products over integrally closed open adic subrings.}
    \end{enumerate}

    \todo[inline]{Completed tensor products}
\end{remark}

\begin{remark}[Graded Manin triples] \label{remark: graded_manin_triples}
    \todo[inline]{Graded Manin triples}
\end{remark}

\begin{remark}[Spectral parameters] \label{remark: spectral_parameters}
    \todo[inline]{Spectral parameters}
\end{remark}

Manin triples naturally form a category wherein morphisms:
    $$\phi: (\a, \a^+, \a^+) \to (\b, \b^+, \b^+)$$
are Lie algebra homomorphisms:
    $$\phi: \a \to \b$$
such that:
    \begin{equation} \label{equation: morphisms_of_manin_triples}
        \begin{gathered}
            \phi( \a^{\pm} ) \subseteq \b^{\pm}
            \\
            (\cdot, \cdot)_{\a} = (\cdot, \cdot)_{\b} \circ (\phi \tensor \phi)
        \end{gathered}
    \end{equation}
\begin{remark}
    The last condition in definition \ref{def: manin_triples} is imposed so that even in the infinite-dimensional setting, the equivalence \eqref{equation: manin_triple_lie_bialgebra_correspondence} between the groupoids of Manin triples and topological Lie bialgebras would hold.
\end{remark}

Next, let us discuss the relationship between Manin triples and topological Lie bialgebras.

If $(\a, \a^+, \a^-)$ is a Manin triple, then we can construct a Lie bialgebra structure on $\a = \a^+ \oplus \a^-$ in the following manner. If we regard the Lie brackets on $\a^{\mp}$ as linear maps:
    $$[\cdot, \cdot]_{\a^{\mp}}: \a^{\mp} \tensor \a^{\mp} \to \a^{\mp}$$
then first of all, dualising yields continuous linear maps:
    \begin{equation} \label{equation: dual_of_lie_brackets}
        \delta^{\pm} := [\cdot, \cdot]_{\a^{\mp}}^*: (\a^{\mp})^{*} \to (\a^{\mp} \tensor \a^{\mp})^*
    \end{equation}
Since $(\a, \a^+, \a^-)$ is a Manin triple, the codomain of $[\cdot, \cdot]_{\a^{\mp}}^*$ lies inside the subspace $(\a^{\mp})^{*} \hattensor (\a^{\mp})^{*} \subseteq (\a^{\mp} \tensor \a^{\mp})^*$ according to definition \ref{def: manin_triples}; by identifying $\a^{\mp} \cong (\a^{\pm})^*$ as topological vector spaces, we then obtain a linear map:
    $$\delta^{\pm}: \a^{\pm} \to \a^{\pm} \hattensor \a^{\pm}$$
Then, by extending the bilinear form $(\cdot, \cdot)_{\a}$ factor-wise to $\a \hattensor \a$, one can then compute the values of $\delta^{\pm}$ by means of identifying:
    \begin{equation} \label{equation: lie_cobrackets_by_duality}
        \left( \delta^{\pm}(x), y_1 \tensor y_2 \right)_{\a \hattensor \a} = \left( x, [y_1, y_2]_{\a^{\mp}} \right)_{\a} \quad, \quad x \in \a^{\pm}, y_1, y_2 \in \a^{\mp}
    \end{equation}
Using the topological Lie bialgebra structures $\delta^{\pm}: \a^{\pm} \to \a^{\pm} \hattensor \a^{\pm}$, we can then construct a topological Lie bialgebra structure on $\a = \a^+ \oplus \a^-$:
    $$\delta := \delta^+ \oplus (-\delta^-): \a \to \a \hattensor \a$$

Conversely, given a topological Lie bialgebra structure $\delta^+: \a^+ \to \a^+ \hattensor \a^+$, one can construct a Manin triple $(\a, \a^+, \a^-)$ with $\a^- := (\a^+)^*, \a := \a^+ \oplus \a^-$, and the non-degenerate and invariant pairing on $\a$ is the canonical one between $\a^+$ and $\a^-$, given by:
    $$(x, \varphi)_{\a} := \varphi(x) \quad, \quad x \in \a^+, \varphi \in \a^-$$
Moreover, $\a^-$ automatically carries an induced topological Lie bialgebra structure $\delta^-$ given by dualising the (continuous) Lie bracket on $\a^+$; there is thus also a topological Lie bialgebra structure on $\a$ given by:
    \begin{equation} \label{equation: classical_double_cobrackets}
        \delta_{\Dr(\a^+)} := \delta^+ \oplus (-\delta^-) = \delta^+ \oplus (\delta^-)^{\cop}
    \end{equation}
As such, the Manin triple $(\a, \a^+, \a^-)$ constructed above is in fact a triple of topological Lie bialgebras; the procedure above that outputs the topological Lie bialgebra $(\a, \delta)$ from the topological Lie sub-bialgebra $(\a^+, \delta^+)$ is commonly known as Drinfeld's \textbf{classical double} construction, and we write:
    $$\a \cong \Dr(\a^+)$$
\begin{remark} \label{remark: embeddings_into_classical_doubles}
    By dualising the construction of the cobracket $\delta := \delta^+ \oplus (-\delta^-)$ as in equation \eqref{equation: classical_double_cobrackets}, we obtain:
        $$[\cdot, \cdot]_{\Dr(\a^+)} = [\cdot, \cdot]_{\a^-} \oplus ( -[\cdot, \cdot]_{\a^+} ) = [\cdot, \cdot]_{\a^-} \oplus [\cdot, \cdot]_{\a^+}^{\op}$$
    (recall that $\delta^{\pm} = [\cdot, \cdot]_{\a^{\mp}}^*$). Therefore, there are the following Lie algebra embeddings into the classical double $\a \cong \Dr(\a^+)$:
        \begin{equation} \label{diagram: embeddings_into_classical_doubles}
            \begin{tikzcd}
            	{\a^+} && {(\a^-)^{\op, \cop} = (\a^+)^{*, \op, \cop}} \\
            	& {\Dr(\a^+)}
            	\arrow[hook, from=1-1, to=2-2]
            	\arrow[hook', from=1-3, to=2-2]
            \end{tikzcd}
        \end{equation}
    By the same reasoning, we see also that:
        $$\Dr(\a^-) \cong \Dr(\a^+)^{\op, \cop}$$
    as the Lie bracket and cobracket on the LHS $\Dr(\a^-)$ are given as follows, respectively:
        $$
            \begin{gathered}
                [\cdot, \cdot]_{\Dr(\a^-)} = [\cdot, \cdot]_{\a^+} \oplus ( -[\cdot, \cdot]_{\a^-} ) = [\cdot, \cdot]_{\a^+} \oplus [\cdot, \cdot]_{\a^-}^{\op}
                \\
                \delta_{\Dr(\a^-)} := \delta^- \oplus (-\delta^+) = \delta^- \oplus (\delta^+)^{\cop}
            \end{gathered}
        $$
\end{remark}

In short, the procedure described above yields us a bijective correspondence:
    \begin{equation} \label{equation: manin_triple_lie_bialgebra_correspondence}
        \left\{ \text{Manin triples $(\a, \a^+, \a^-)$} \right\} \leftrightarrows \left\{ \text{Topological Lie bialgebra structures $(\a^+, \delta^+)$} \right\}
    \end{equation}
wherein the forward map is given by $\delta^+ := [\cdot, \cdot]^*_{\a^-}$ while the backward map sends $(\a^+, \delta^+)$ to the canonical \say{classical double} Manin triple $( \Dr(\a^+), \delta_{\Dr(\a^+)} )$ constructed by means of equation \eqref{equation: classical_double_cobrackets}. This bijection can be enhanced to an equivalence of categories in a straightforward manner.